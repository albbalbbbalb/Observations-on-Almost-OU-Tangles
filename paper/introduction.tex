\section{Introduction}

In the article \citep{barnatan2020tangles} we are introduced with some Knot Theoretic structures including tangles, tangle diagrams, braids, most importantly the Over then Under form of a tangle, or OU form for short, followed by a proposed algorithm for bringing a tangle diagram into an OU form, and some results relating to this form.

Although the results presented in the previously mentioned article are quite broad and general, in our article for the sake of simplicity we restrict ourselves to the special case of a tangle with a single strand, and focus on what properties or patterns we can discover by examining 1-tangles under the process of the OU form algorithm. We shall see that our choice of tangles to study will prove to be erroneous, and that we will encounter a level of complexity that we did not bargain for. Specifically, the only tangle diagrams that behave nicely under the algorithm are trivial, and even then there is a set of trivial diagrams which do not behave nicely. Furthermore the algorithm in the case of misbehavior never terminates, and the diagrams that are produced at each iteration slowly tend to the realm of the wild. 

Given the level of research intended here, before we begin we must draw out the definitions and concepts necessary to understand, reason, and ask further questions about this topic. We settle with a working definition, and attempt to introduce only the bare minimum. Should the reader be interested in a more comprehensive overview of Knot Theory, they are directed to the works \citep{adams_1994} and \citep{lickorish_1997}.

